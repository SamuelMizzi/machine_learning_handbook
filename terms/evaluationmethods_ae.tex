\chapter{Evaluation Methods}
\label{ch:evaluation-methods}
\index{evaluation methods|(}

Although modern machine learning is exceptionally diverse and multidisciplinary, what unifies the field is standard experimental evaluation methodologies that are adopted based on the task at hand~\citep{Kibler1988}.
%
Evaluation is involved both in model selection, i.e.\ during hyperparameter tuning, and in comparing multiple models against each other or a random baseline.
Ideally, evaluation is aimed at interpreting a model's behaviour and analysing its errors~\citep{Doshi2017}.

\citet{Japkowic2011} outline four essentials of evaluation methodology: a test set, performance measures, error estimation, and statistical significance tests.

When evaluating machine learning models, it is common to rely on some annotated \emph{gold standard}\index{evaluation methods!gold standard} dataset that serves as a proxy of real-world phenomena~\citep{Manning2008}.
%
Examples of such datasets can be found in centralised collections such as the UCI Machine Learning repository~\citep{Dua2017} or online communities like Kaggle\footnote{\url{https://www.kaggle.com}}.
However, one needs to be aware that a dataset
may not represent the whole domain~\citep{Japkowic2011},
might have been constructed using vague class definitions,
or contains annotation artefacts~\citep{Hand2006}.

While there is an assumption that training and testing sets come from the same distribution in order for a model to work as expected~\citep{Japkowic2011}, it is essential that the model is tested on previously unseen data.
Otherwise, it yields overoptimistic results~\citep{Kibler1988}.
%
Error estimation techniques such as a held-out\index{cross-validation!holdout} set or cross-validation are detailed
% \nameref{ch:cross-validation},
on page~\pageref{ch:cross-validation}.

\section{Performance Measures}
\citet{Stapor17} notes that according to the \emph{no free lunch theorem}\index{no free lunch theorem}~\citep{Wolpert1996}, it is unreasonable to expect that one classifier is generally better than another since there will always be problems where one outperforms the other and vice versa.
Evaluation is also highly task-specific and selecting an appropriate performance measure should be done with the end goal in mind~\citep{goodfellow2016deeplearning}. For example, a wake-word detection component of a digital assistant might be penalised more for missing the trigger word (a minority class among continuous speech), but a pretrained image classifier later used to solely recognise cats can ignore other smaller classes and there is no need to specifically evaluate those.

\subsection{Automatic Evaluation Metrics}
For supervised classification, the simplest measure is \emph{accuracy}\index{confusion matrix!accuracy} which is the ratio of correctly predicted examples, and its reverse metric \emph{error rate}\index{evaluation methods!error rate}, i.e.\ the fraction of errors.
%
More fine-grained classification metrics are described
% in \nameref{ch:confusion-matrix},
on page~\pageref{ch:confusion-matrix}.

For machine-learned ranking algorithms
% \index{machine-learned ranking}
that infer an ordered list of multiple outputs, standard metrics can be reported \emph{at N}\index{evaluation methods!metrics at N}~\citep{Herlocker2004},
e.g.\
accuracy at 5 (Acc@5) illustrating a relaxed accuracy that looks for a match among the top five outputs for each example.

For cluster analysis, despite it being unsupervised, typical evaluation metrics still require some labelled gold standard dataset.
The simplest measure is \emph{purity}\index{evaluation methods!purity} that indicates the degree
a model is able to cluster together instances of the same ground truth class (Equation~\ref{eq:purity}):
\begin{equation}
    \label{eq:purity}
    p(\Omega, C) = \frac{1}{N} \sum_{k} \max_j \lvert \omega_k \cap c_j \rvert,
\end{equation}
where $\Omega=\{\omega_1,\ldots\omega_k\}$ is the set of predicted clusters, $C$ is the set of reference classes, $N$ is the size of the test set~\citep{Manning2008}.
For other metrics used in clustering see, e.g.~\citet{Friedman2001}.

For regression methods, a common metric to report is the
\emph{root mean squared error} (RMSE)\index{evaluation methods!root mean squared error (RMSE)} that characterises the discrepancy between predicted and true values (Equation~\ref{eq:rmse}):
\begin{equation}
    \label{eq:rmse}
    RMSE(f) = \sqrt{\frac{1}{N} \sum_i^N (f(x_i) - y_i)^2},
\end{equation}
where $f(x_i)$ is a prediction for input $x_i$, $y_i$ is the target value, and $N$ is the test set size.
\citet{Harrell2015} overviews other regression metrics.

\subsection{Manual Evaluation Metrics}
It is not always possible to use the aforementioned automatic metrics to assess a machine learning model in a task-specific context, e.g.\ when there are no ground truth data available or there are yet no reliable metrics correlating with human judgement or other real-world phenomena.
Therefore, \emph{manual}\index{evaluation methods!evaluation!manual} evaluation is sometimes required and is typically done either by experts or via crowdsourcing\index{evaluation methods!crowdsourcing} when raters are performing small paid evaluation tasks~\citep{Alonso2008} using specialised platforms such as Amazon Mechanical Turk\footnote{\url{https://www.mturk.com}}.
For example, assessors can be provided with Likert scales\index{evaluation methods!Likert scale}~\citep{Likert1932} to rate various facets of a model's output, e.g.\ fluency of machine translation on a scale from 0 to 5;
or they can be asked to rank outputs according to some aspect, allowing comparing multiple models.
%
There have been attempts to replicate such judgements automatically, e.g.\
using \emph{adversarial} evaluation\index{evaluation methods!evaluation!adversarial}~\citep{martin2018speech} that
% , in the spirit of the Turing test,
involves training Generative Adversarial Networks\index{generative adversarial networks}~\citep{Goodfellow2014} to distinguish between human and machine-generated outputs.

Another way to overcome the lack of metrics and data is to shift the task on which a model is evaluated, i.e.\ conduct \emph{extrinsic}\index{evaluation methods!evaluation!extrinsic} evaluation~\citep{Belz2008} that commonly involves assessment of a model's performance in a ``downstream'' application rather than the training task, as opposed to \emph{intrinsic}\index{evaluation methods!evaluation!intrinsic} evaluation.
For instance, if the goal is to evaluate a feature reduction mechanism, one can measure its effect on the accuracy of sentiment classification~\citep{Schnabel2015}.
%
An example of manual extrinsic evaluation is A/B studies\index{evaluation methods!A/B testing}~\citep{Hofmann2016} aimed to measure a change in user experience before quantitatively and after deploying a new machine learning model.
A/B testing consists of dividing users two groups A and B and comparing them against each other according to some metrics.
%
A/B tests can be employed to investigate the average time required for users to find the desired webpage depending on the search algorithm~\citep{Manning2008}
% or how quickly users can book a flight using different conversational agents~\citep{Jurafsky2009};
or the average time spent by users on the website after introducing a new recommendation system~\citep{Davidson2010}.

\section{Statistical Significance Testing}\index{evaluation methods!statistical significance testing|(}

Whenever a new algorithm is proposed, it needs to be compared against either a random baseline or previously published results for the task and dataset at hand.
However, often improvement is rather incremental~\citep{Hand2006}, so it might not be evident if it is of any importance.
To decide whether a change is significant, \emph{statistical significance tests} are encouraged, at least in case of scarce testing data~\citep{Drummond2010}.

Details about general statistical tests can be found in~\citep{Urdan2011}, whereas~\citet{Japkowic2011} provide recommendations on the correct use of such tests for classifier comparison.
In evaluating machine learning techniques, one needs to be cautious when using standard parametric\index{evaluation methods!statistical significance testing!parametric} tests such as the $t$-test~\citep{Urdan2011} because of the assumptions expected to be met, e.g.\ equal variance of obtained metrics~\citep{Japkowic2011}. In contrast, \emph{non-parametric}\index{evaluation methods!statistical significance testing!non-parametric} tests do not impose any restriction on the population distribution which makes them more applicable, although less powerful~\citep{Colquhoun1971}.
%
Below is an illustration of applying a non-parametric test.

\subsection{An Illustration of Statistical Significance Testing}
\begin{margintable}
    \begin{tabular}{llrr}
        \toprule
        & & \multicolumn{2}{c}{\textbf{B}} \\
        & & Correct & Incorrect \\ \midrule
        \textbf{A} & Correct & $CC = 1200$ & $CI = 10$ \\
        & Incorrect & $IC = 25$   & $II = 8$  \\
        \bottomrule
    \end{tabular}
    \caption{Contingency table breaking down predictions of two classifiers $A$ and $B$ into different cases based on comparison against the ground truth labels (correct or incorrect).}
    \label{tab:eval-mcnemar-table}
\end{margintable}

Given two trained classifiers $A$ and $B$ and a set of ground truth labels of size $N$ against which the predictions can be compared, it is possible to construct a $2\times 2$ contingency table such as Table~\ref{tab:eval-mcnemar-table}.
Assuming that accuracy is selected as a performance metric, it can be seen that $N = 1243$, the accuracy of model $A$ is $(CC + CI) / N = 1210 / 1243 = 0.973$, while for model $B$ it is $0.986$.
Based on the accuracy alone, the improvement of classifier $B$ appears rather small, so one would want to report statistical significance.
A \emph{null hypothesis}\index{evaluation methods!statistical significance testing} $H_0$ is formulated that there is no difference in accuracy between classifiers, whereas the alternative hypothesis $H_1$ states that these are significantly different.
A \emph{significance level} $\alpha$ representing the probability of wrongly rejecting $H_0$ is specified, e.g.\ 0.05.
Supposing that it has been found that assumptions for parametric tests do not hold, a non-parametric alternative is used, e.g.\ \emph{McNemar's test}\index{evaluation methods!statistical significance testing!McNemar's test}~\citep{Mcnemar1947} with the following statistic:
\begin{equation}
    \label{eq:mcnemar}
    \chi^2_{MC} = \frac{(\lvert IC - CI  \rvert - 1)^2}{IC + CI},
\end{equation}
where $IC$ and $CI$ correspond to the cases described in Table~\ref{tab:eval-mcnemar-table}.
%
The $\chi^2_{MC}$ statistic from Equation~\ref{eq:mcnemar} is compared against the $\chi^2$ distribution table value (or a binomial distribution in case of $CI + IC < 20$).
If $\chi^2_{MC}$ is greater than the table value, $H_0$ is rejected.

For the example in Table~\ref{tab:eval-mcnemar-table}, $\chi^2_{MC}$ is 5.6, which is greater than the critical $\chi^2$ value for the selected $\alpha=0.05$ (3.841), therefore, it is said that the improvement in accuracy is significant.
\index{evaluation methods!statistical significance testing|)}

In conclusion, each component of a machine learning evaluation framework
% , i.e.\ selecting a testing dataset and performance measures, conducting error estimation and statistical significance tests,
requires careful consideration with respect to the end goal, however, rigorous evaluation can pave the way to interpretable and more reliable machine learning systems~\citep{Doshi2017}.

\index{evaluation methods|)}